\documentclass[times, utf8, seminar]{fit}

%\batchmode
%\usepackage{booktabs}
\usepackage{listings}
\usepackage{longtable}
\usepackage{xcolor}
\usepackage{float}
\usepackage{enumitem}
\usepackage{hyperref}
\usepackage{enumerate}
\usepackage{graphicx}
\usepackage{etoolbox}
\usepackage{datetime}

\begin{document}

\title{Open source software in Bosnia and Herzegovina}

\author{Ernad Husremović}
\brindex{DL 2792}
\verzija {0.0.1}

\mentor{Iris Memić}

\maketitle

\tableofcontents

\listoftables
\listoffigures

% abstract begin
\begin{abstract}

bla bla 

\keywords{open source software, OSS, multiplatform software}
\end{abstract}

% abstract end

\chapter{Introduction}

\section{Open source software}

Open source software (OSS) refers to the software that is released with source code. The core philosophy of open source is  sharing.

OSS Community is the place where the sharing takes place.

OSS projects are typically collaborative effort in which programmers create and share the code within the community.

\subsection{Open source definition}
Bruce Perens was the first person who announced "Open Source" to the world in 1998, and is the creator of the Open Source Definition\footnote{\url{http://www.opensource.org/docs/osd}}, the manifesto of the Open Source movement and the rules required for a software license to be considered "Open Source"\citep{web:perens}. Today, Perens works as a leader in the Open Source and Free Software community. He advises many large companies and several national governments on issues related to Open Source. Perens' main policy areas regarding Open Source are:

\begin{center}
\emph{\large{Freedom to create, distribute, and use open source software (OSS).}}
\end{center}


\subsection{Open source vs. Free Software}

In his Open Source work, Perens is standing on the shoulders of giants: in particular \emph{Richard Stallman}, founder of the Free Software movement\footnote{\url{http://www.gnu.org/philosophy/free-software-intro.html}} in the early 1980s. Perens positions Open Source as a different way of talking about Free Software, intended for a different audience - business people and those who would be more receptive to an economic and pragmatic argument than to Stallman's focus on Freedom\citep{web:perens}.

Perens believes that Open Source and Free Software are a single movement rather than two conflicting ones. Perens believes that promotion of Open Source should not deprecate Stallman or his philosophy.

Stallman himself understands but does not entirely accept Perens' slant on using the language of Open Source to promote Free Software. This is Stallman's statement:

\begin{quotation}
\emph{Free software and Open Source seem quite similar, if you look only at their software development practices. At the philosophical level, the difference is extreme. The Free Software Movement is a social movement for computer users' freedom. The Open Source philosophy cites practical, economic benefits. A deeper difference cannot be imagined.}
\end{quotation}

Despite their differences, Perens and Stallman maintain a good relationship and work together frequently.
\section{Open source licences}

Software license determines constraints on software usage and redistribution. Two main categories are proprietary licenses and free and open source (FOSS) licenses.

Later, FOSS licensed fall in two main categories:
\begin{itemize}
  \item copyleft licenses
  \item permisive licenses 
\end{itemize} 

Copyleft licenses are focused on preserving freedom to study, modify and use software by all users. Consequently they force authors to release modifications under the same terms. It is the reason they are sometimes referred to as "viral licenses".

Permissive licenses have minimal requirements on preserving openness to subsequent users. The main characteristic of permissive licenses is the possibility to take the source code and merge with closed source software. The author of modificatios decides wheter they will be released as closed software or as open source software. The possibility to mix closed and open source code permissive licenses tag "business friendly".

GNU General Public License\footnote{\url{http://www.gnu.org/copyleft/gpl.html}} is the main representant of copyleft licenses.

MIT\footnote{\url{http://www.opensource.org/licenses/mit-license.html}} and BSD licenses\footnote{\url{http://www.opensource.org/licenses/BSD-3-Clause/}} are most popular permissive licenses.

\section{OSS business models}

There are many business models based on OSS\footnote{\url{http://www.opensourcestrategies.org/}}. We point out these two models: 
\begin{itemize}
  \item service based
  \item open core 
\end{itemize}


Service based OSS business offers its expertise and support to its customers. The software itself is free and open source and such not the subject of bussiness offer.

"Open core" model offers basic functionalities as open source, but advanced (enterprised) features as closed source.

There is much criticism on open core model. The opponents stands that "open core" is much like a variant of trial versions of closed software.

Choosing of specific model is highly correlated by the class of software.  For example, "open core" model is largely represented between ERP\footnote{Enterprise Resource Planning} software.  

\section{Piracy}

We use the word "piracy" to describe the ubiquitous, increasingly digital practices of copying that fall outside the boundaries of copyright law. The term blurs, and is often used intentionally to blur, important distinctions between types of uncompensated use. These range from the clearly illegal, such as commercial-scale, unauthorized copying for resale, to disputes over the boundaries of fair use and first sale as applied to digital goods, to the wide range of practices of personal copying that have traditionally fallen below the practical threshold of enforcement\citep{mediapiracy}


\subsection{BSA's partial way of piracy accounting}


\begin{quotation}
\emph{In 2009, more than four out of 10 software programs installed on personal computers around the world were stolen, with a commercial value of more than \$51 billion. Unauthorized software can manifest in otherwise legal businesses that buy too few software licenses, or overt criminal enterprises that sell counterfeit copies of software programs at cut-rate prices, online or offline.}

\emph{However, the impact of software piracy goes beyond revenues lost to the software industry, starving local software distributors and service providers of spending that creates jobs and generates much-needed tax revenues for governments around the world.}\citep{bsapiracyimpact}
\end{quotation}\\

We consider this way of piracy accounting partial because it doesn't count positive impact of piracy on market position of top-tier (closed) vendors. As we know that BSA and similar organizations mainly presents interest of those vendors, this point of view on this subject is understandable.  

\subsection{Software "Lock-in"}

"Lock-in" occurs when the costs of leaving a particular software environment are high - whether because switching would require significant repurchasing of software, or because the use of less common standards is disadvantageous, or simply due to costs of retraining. For near-monopolies such as Microsoft in the operating systems and office software markets, network effects reinforce market power and increase the value of their products.

Lock-in effects, in turn, ensure that customers are less likely to switch to competitors.

\subsection{Software piracy as an instrument for market dominance}  

In software markets, network effects refer to contexts in which the value of software rises with the size of the installed base. The more widely used a piece of software or software service, the more it becomes a de facto standard that shapes user decisions about adoption and investment. Platform technologies such as operating systems exhibit strong network effects because a popular platform will foster a rich secondary market in applications and services, which in turn increases the platform’s value. 

As BSA piracy figures indicate\footnote{Business software alliance}, these dynamics in emerging economies are primarily a function of pirated-software adoption, not legal adoption. Piracy, in effect, has allowed the major vendors to dominate low- and middle-income markets (or, as they develop, market segments within them) that they have little financial incentive to serve. 

Perhaps most important for market-dominating firms, piracy acts as a \textbf{barrier to entry for competition, especially "free" open-source alternatives} that have no upfront licensing costs. When these emerging markets begin to grow, as most did in the last decade, piracy ensures they do so along paths shaped by the powerful network and lock-in effects associated with the market leaders.

\subsection{Software piracy full accounting}

These factors should figure in any full accounting of the \textbf{costs} and \textbf{benefits} of software piracy for top-tier vendors. Top-tier vendors have established and maintained their dominant positions in emerging markets through piracy, often prior to or in the absence of significant local investment. Any losses they incur at the margins of the consumer and business markets in those countries should be weighed against the value of maintaining their dominant positions. For near-monopolies, we would argue that this value is very high. For vendors working in highly competitive markets or selling products that do not function as standards or platforms, that value is clearly lower.\citep{mediapiracy}


\chapter{Conclusion}

% -------------------------------------------------
\bibliography{literatura}
\bibliographystyle{fit}

% -------------------------------------------------
\appendix

\chapter{Software toolset}
\begin{enumerate}
  \item Mac OS X 10.6.8
  \item mvim, vim tekst editor ver 7.3
  \item MacTex - pdfTeX 3.1415926-2.3-1.40.12 (TeX Live 2011)
\end{enumerate}

\chapter{Notes}

\begin{itemize}
  \item Document source code \url{https://github.com/hernad/FIT_ENG/blob/master/seminarski/OS_BiH.tex}
  \item Document in PDF format \url{https://github.com/hernad/FIT_ENG/raw/master/seminarski/OSS_BiH.pdf}
\end{itemize}

\end{document}
