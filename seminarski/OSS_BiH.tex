\documentclass[times, utf8, seminar]{fit}

%\batchmode
%\usepackage{booktabs}
\usepackage{listings}
\usepackage{longtable}
\usepackage{xcolor}
\usepackage{float}
\usepackage{enumitem}
\usepackage{hyperref}
\usepackage{enumerate}
\usepackage{graphicx}
\usepackage{etoolbox}
\usepackage{datetime}

\begin{document}

\title{Open source software in Bosnia and Herzegovina}

\author{Ernad Husremović}
\brindex{DL 2792}
\verzija {0.0.1}

\mentor{Iris Memić}

\maketitle

\tableofcontents

\listoftables
\listoffigures

% abstract begin
\begin{abstract}

bla bla 

\keywords{open source software, OSS, multiplatform software}
\end{abstract}

% abstract end

\chapter{Introduction}

\section{Piracy}

We use the word "piracy" to describe the ubiquitous, increasingly digital practices of copying that fall outside the boundaries of copyright law. The term blurs, and is often used intentionally to blur, important distinctions between types of uncompensated use. These range from the clearly illegal, such as commercial-scale, unauthorized copying for resale, to disputes over the boundaries of fair use and first sale as applied to digital goods, to the wide range of practices of personal copying that have traditionally fallen below the practical threshold of enforcement\citep{mediapiracy}


\section{"Lock-in"}

"Lock-in" occurs when the costs of leaving a particular software environment are high - whether because switching would require significant repurchasing of software, or because the use of less common standards is disadvantageous, or simply due to costs of retraining. For near-monopolies such as Microsoft in the operating systems and office software markets, network effects reinforce market power and increase the value of their products.

Lock-in effects, in turn, ensure that customers are less likely to switch to competitors.

As BSA piracy figures indicate\footnote{Business software alliance}, these dynamics in emerging economies are primarily a function of pirated-software adoption, not legal adoption.

Piracy, in effect, has allowed the major vendors to dominate low- and middle-income markets (or, as they develop, market segments within them) that they have little financial incentive to serve.

Perhaps most important for market-dominating firms, piracy acts as a barrier to entry for competition, especially "free" open-source alternatives that have no upfront licensing costs.

When these emerging markets begin to grow, as most did in the last decade, piracy ensures they do so along paths shaped by the powerful network and lock-in effects associated with the market leaders.\citep{mediapiracy}


\chapter{Conclusion}

% -------------------------------------------------
\bibliography{literatura}
\bibliographystyle{fit}

% -------------------------------------------------
\appendix

\chapter{Software toolset}
\begin{enumerate}
  \item Mac OS X 10.6.8
  \item mvim, vim tekst editor ver 7.3
  \item MacTex - pdfTeX 3.1415926-2.3-1.40.12 (TeX Live 2011)
\end{enumerate}

\chapter{Notes}

\begin{itemize}
  \item Document source code \url{https://github.com/hernad/FIT_ENG/blob/master/seminarski/OS_BiH.tex}
  \item Document in PDF format \url{https://github.com/hernad/FIT_ENG/raw/master/seminarski/OSS_BiH.pdf}
\end{itemize}

\end{document}
